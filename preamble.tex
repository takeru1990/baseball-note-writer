\documentclass[a5paper]{jsarticle}
\usepackage{setspace}
\usepackage{multirow}
\usepackage{multicol}
\usepackage[top=10truemm,bottom=10truemm,left=13truemm,right=13truemm]{geometry}

\setstretch{0.9}
\setlength{\parskip}{1ex plus 0.5ex minus 0.2ex}

\usepackage{plext}
\usepackage{cancel}
\usepackage{color}

\begin{document}
\title{\HUGE{野球ノート}}
\date{\flushright{2003年5月18日 - 2004年6月12日}}
\maketitle
\thispagestyle{empty}
\newpage
%\twocolumn%ここまで表紙の情報

\setcounter{page}{1}%次のページを1ページとする

%%%%%%%%%%%%%%%%%%%%%%%ここからテンプレート
\begin{center}
	{\bfseries \LARGE %team name
		\begin{tabular}{ccc}
			%学校名 &-&%学校名 \\
			%複数試合の時は&-&%その分記入する 
		\end{tabular}
		}\noindent \rule{\columnwidth}{0.1mm} \\
	\begin{tabular}{cccccc}
		\multirow{%試合数}{*}{結果}  & %先攻得点 &-& %後攻側得点  &:&%自分の勝敗\\
		  & %先攻得点 &-& %後攻得点  &:&%自分の勝敗 
	\end{tabular}
\end{center}
%試合内容
\medskip \noindent \fbox{学んだこと}
\begin{itemize}
	\item %学んだことを
	\item %箇条書きにまとめる
\end{itemize}
\begin{flushright}(%ヒヅケ)\end{flushright}
\newpage
%%%%%%%%%%%%%%%%%%%%ここまでテンプレート,必要分繰り返す

\end{document}
